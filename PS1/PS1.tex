\documentclass{article}
 \usepackage{booktabs}
 \usepackage{graphicx}
 \usepackage{fancyhdr}
  \usepackage[margin=1in]{geometry}
 \setlength{\parindent}{0pt}
\setcounter{page}{1} 
\pagestyle{fancy}
 \fancyhead{}
 \fancyhf{}
 \renewcommand{\headrulewidth}{0pt}
 \fancyfoot{}
 \fancyfoot[R]{\thepage}
 \begin{document}
\textbf{Part 1 - No Skilled Funds} \\\textbf{Figure 1 : }The figure below depicts the distribution of simulated T-statistics associated with alpha. Alpha is estimated with simulated returns which are generated such that true alpha is equal to zero for each of 1000 funds/returns.
\begin{center}
 \begin{figure}[h!]
 \includegraphics[scale=0.8]{plot1.pdf}\label{fig:plot1}
 \end{figure}
 \end{center}
\clearpage 
\textbf{Figure 2 : }The figure below depicts the distribution of simulated p-values associated with alpha. Alpha is estimated with simulated returns which are generated such that true alpha is equal to zero for each of 1000 funds/returns.
\begin{center}
 \begin{figure}[h!]
 \includegraphics[scale=0.8]{plot2.pdf}\label{fig:plot2}
 \end{figure}
 \end{center}
\clearpage 
\textbf{Part 2 - Some Skilled Funds} \\\textbf{Figure 3 : }The figure below depicts the distribution of simulated alpha estimates. Returns are simulated such that a fraction $\lambda$ of the funds/returns are truly skilled ($\alpha =5$\% per annum).
\begin{center}
 \begin{figure}[h!]
 \includegraphics[scale=0.8]{plot3.pdf}\label{fig:plot3}
 \end{figure}
 \end{center}
\clearpage 
\textbf{Figure 4 : }The figure below depicts the distribution of simulated T-statistics associated with alpha. Returns are simulated such that a fraction $\lambda$ of the funds/returns are truly skilled ($\alpha =5$\% per annum).
\begin{center}
 \begin{figure}[h!]
 \includegraphics[scale=0.8]{plot4.pdf}\label{fig:plot4}
 \end{figure}
 \end{center}
\clearpage 
\textbf{Figure 5 : }The figure below depicts the distribution of simulated p-values associated with alpha. Returns are simulated such that a fraction $\lambda$ of the funds/returns are truly skilled ($\alpha =5$\% per annum).
\begin{center}
 \begin{figure}[h!]
 \includegraphics[scale=0.8]{plot5.pdf}\label{fig:plot5}
 \end{figure}
 \end{center}
\clearpage 
\textbf{Tables 1-4 : }The below tables display the perentage of funds falling into each true vs estimated skill category. Returns are simulated such that a fraction $\lambda$ of the funds/returns are truly skilled ($\alpha =5$\% per annum).  I am not sure why the tables are not appearing on the same page; I need to experiment with this more.\begin{center}
\begin{table}[h]
\centering
\caption{Lambda=.1}
\begin{tabular}{llrr}
\toprule
           &   & \multicolumn{2}{l}{Estimated Skill} \\
           &   &               Y &     N \\
\midrule
True Skill & Y &             0.9 &   9.1 \\
           & N &             2.1 &  87.9 \\
\bottomrule
\end{tabular}
\end{table}
\end{center}
\begin{center}
\begin{table}[h]
\centering
\caption{Lambda=.25}
\begin{tabular}{llrr}
\toprule
           &   & \multicolumn{2}{l}{Estimated Skill} \\
           &   &               Y &     N \\
\midrule
True Skill & Y &             2.2 &  22.8 \\
           & N &             1.9 &  73.1 \\
\bottomrule
\end{tabular}
\end{table}
\end{center}
\begin{center}
\begin{table}[h]
\centering
\caption{Lambda=.5}
\begin{tabular}{llrr}
\toprule
           &   & \multicolumn{2}{l}{Estimated Skill} \\
           &   &               Y &     N \\
\midrule
True Skill & Y &             5.6 &  44.4 \\
           & N &             0.8 &  49.2 \\
\bottomrule
\end{tabular}
\end{table}
\end{center}
\begin{center}
\begin{table}[h]
\centering
\caption{Lambda=.75}
\begin{tabular}{llrr}
\toprule
           &   & \multicolumn{2}{l}{Estimated Skill} \\
           &   &               Y &     N \\
\midrule
True Skill & Y &             8.0 &  67.0 \\
           & N &             0.4 &  24.6 \\
\bottomrule
\end{tabular}
\end{table}
\end{center}
\end{document}
