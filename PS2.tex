\documentclass{article}
 \usepackage{booktabs}
 \usepackage{graphicx}
 \usepackage{fancyhdr}
  \usepackage[margin=1in]{geometry}
 \setlength{\parindent}{0pt}
\setcounter{page}{1} 
\pagestyle{fancy}
 \fancyhead{}
 \fancyhf{}
 \renewcommand{\headrulewidth}{0pt}
 \fancyfoot{}
 \fancyfoot[R]{\thepage}
 \begin{document}
\textbf{Part 1 - No Skilled Funds} \\\textbf{Table 1 : }The table depicts the average alpha T-statistic at each rank (1-15 and 985-100) across 100 boot-stapped time series, constructed so that the true alpha is equal to 0 for all funds.\begin{center}
\begin{table}[h]
\centering
\caption{Average T-statistic across bootstrap runs, by rank}
\begin{tabular}{lrr}
\toprule
{} &  Rank &  Tstat \\
\midrule
{} &     1 &  -3.12 \\
{} &     2 &  -2.88 \\
{} &     3 &  -2.74 \\
{} &     4 &  -2.64 \\
{} &     5 &  -2.57 \\
{} &     6 &  -2.51 \\
{} &     7 &  -2.45 \\
{} &     8 &  -2.40 \\
{} &     9 &  -2.35 \\
{} &    10 &  -2.31 \\
{} &    11 &  -2.27 \\
{} &    12 &  -2.24 \\
{} &    13 &  -2.21 \\
{} &    14 &  -2.18 \\
{} &    15 &  -2.16 \\
{} &   985 &   2.13 \\
{} &   986 &   2.16 \\
{} &   987 &   2.18 \\
{} &   988 &   2.22 \\
{} &   989 &   2.24 \\
{} &   990 &   2.28 \\
{} &   991 &   2.31 \\
{} &   992 &   2.36 \\
{} &   993 &   2.40 \\
{} &   994 &   2.46 \\
{} &   995 &   2.51 \\
{} &   996 &   2.58 \\
{} &   997 &   2.66 \\
{} &   998 &   2.75 \\
{} &   999 &   2.91 \\
{} &  1000 &   3.15 \\
\bottomrule
\end{tabular}
\end{table}
\end{center}
\textbf{Table 2 : }The table depicts the average 95th and 5th alpha T-statistic across 100 boot-stapped time series, constructed so that the true alpha is equal to 0 for all funds.\begin{center}
\begin{table}[h]
\centering
\caption{Average 95th and 5th percentile T-statistic across bootstrap runs}
\begin{tabular}{lrr}
\toprule
{} &  95th Percentile &  5th Percentile \\
\midrule
{} &             1.61 &           -1.63 \\
\bottomrule
\end{tabular}
\end{table}
\end{center}
\newpage\textbf{Figure 1 : }Data are generated such that there are no truly skilled funds. The figure shows the CDF of alpha T-statistics, estimated using bootstrapped (orange) and actual (blue) data. 95th and 5th bootsrap percentiles are given by the green and red dashed lines, respectively.  
\begin{center}
 \begin{figure}[h!]
 \hspace{1cm}\includegraphics[scale=1.0]{plot1.pdf}\label{fig:plot1}
 \end{figure}
 \end{center}
\clearpage 
\textbf{Figure 2 : }Data are generated such that there are no truly skilled funds. The second figure shows the distributions of estimated 95th (orange) and 5th (blue) percentile alpha T-statistics across bootstrap samples.
\begin{center}
 \begin{figure}[h!]
 \hspace{1cm}\includegraphics[scale=0.9166666666666667]{plot2.pdf}\label{fig:plot2}
 \end{figure}
 \end{center}
\clearpage 
\textbf{Part 2 - Some Skilled Funds} \\\textbf{Figure 3 : }Figures 3-6 show the cumulative distributions of estimated alpha T-statistics at each of four levels of lambda (.1, .25, .5, .75), the fraction of truly skilled funds.  In each figure, lambda is given in the title.  For each lambda, the first panel conducts the procedure so that the true alpha of the underlying distribution is 1\%.  The second and third use alpha =2.5\% and 5\%, respectively.  95th and 5th bootsrap percentiles are given by the green and red dashed lines, respectively. The CDF of alpha T-statistics is estimated using bootstrapped (orange) and actual (blue) data.  In the below, $\lambda=.1$.
\begin{center}
 \begin{figure}[h!]
 \includegraphics[scale=0.7]{plot1a.pdf}\label{fig:plot1a}
 \end{figure}
 \end{center}
\clearpage 
\textbf{Figure 4 : }Figures 3-6 show the cumulative distributions of estimated alpha T-statistics at each of four levels of lambda (.1, .25, .5, .75), the fraction of truly skilled funds.  In each figure, lambda is given in the title.  For each lambda, the first panel conducts the procedure so that the true alpha of the underlying distribution is 1\%.  The second and third use alpha =2.5\% and 5\%, respectively.  95th and 5th bootsrap percentiles are given by the green and red dashed lines, respectively. The CDF of alpha T-statistics is estimated using bootstrapped (orange) and actual (blue) data.  In the below, $\lambda=.25$.
\begin{center}
 \begin{figure}[h!]
 \includegraphics[scale=0.7]{plot2a.pdf}\label{fig:plot2a}
 \end{figure}
 \end{center}
\clearpage 
\textbf{Figure 5 : }Figures 3-6 show the cumulative distributions of estimated alpha T-statistics at each of four levels of lambda (.1, .25, .5, .75), the fraction of truly skilled funds.  In each figure, lambda is given in the title.  For each lambda, the first panel conducts the procedure so that the true alpha of the underlying distribution is 1\%.  The second and third use alpha =2.5\% and 5\%, respectively.  95th and 5th bootsrap percentiles are given by the green and red dashed lines, respectively. The CDF of alpha T-statistics is estimated using bootstrapped (orange) and actual (blue) data.  In the below, $\lambda=.5$.
\begin{center}
 \begin{figure}[h!]
 \includegraphics[scale=0.7]{plot3a.pdf}\label{fig:plot3a}
 \end{figure}
 \end{center}
\clearpage 
\textbf{Figure 6 : }Figures 3-6 show the cumulative distributions of estimated alpha T-statistics at each of four levels of lambda (.1, .25, .5, .75), the fraction of truly skilled funds.  In each figure, lambda is given in the title.  For each lambda, the first panel conducts the procedure so that the true alpha of the underlying distribution is 1\%.  The second and third use alpha =2.5\% and 5\%, respectively.  95th and 5th bootsrap percentiles are given by the green and red dashed lines, respectively. The CDF of alpha T-statistics is estimated using bootstrapped (orange) and actual (blue) data.  In the below, $\lambda=.75$.
\begin{center}
 \begin{figure}[h!]
 \includegraphics[scale=0.7]{plot4a.pdf}\label{fig:plot4a}
 \end{figure}
 \end{center}
\clearpage 
\textbf{Figure 7 : }Figures 7-10 show the distributions of estimated 95th (orange) and 5th (blue) percentile alpha T-statistics across bootstrap runs at each of four levels of lambda (.1, .25, .5, .75), the fraction of truly skilled funds.  In each figure, lambda is given in the title.  For each lambda, the first panel conducts the procedure so that the true alpha of the underlying distribution is 1\%.  The second and third use alpha =2.5\% and 5\%, respectively.  In the below, $\lambda=.1$.
\begin{center}
 \begin{figure}[h!]
 \includegraphics[scale=0.7]{plot1b.pdf}\label{fig:plot1b}
 \end{figure}
 \end{center}
\clearpage 
\textbf{Figure 8 : }Figures 7-10 show the distributions of estimated 95th (orange) and 5th (blue) percentile alpha T-statistics across bootstrap runs at each of four levels of lambda (.1, .25, .5, .75), the fraction of truly skilled funds.  In each figure, lambda is given in the title.  For each lambda, the first panel conducts the procedure so that the true alpha of the underlying distribution is 1\%.  The second and third use alpha =2.5\% and 5\%, respectively.  In the below, $\lambda=.25$.
\begin{center}
 \begin{figure}[h!]
 \includegraphics[scale=0.7]{plot2b.pdf}\label{fig:plot2b}
 \end{figure}
 \end{center}
\clearpage 
\textbf{Figure 9 : }Figures 7-10 show the distributions of estimated 95th (orange) and 5th (blue) percentile alpha T-statistics across bootstrap runs at each of four levels of lambda (.1, .25, .5, .75), the fraction of truly skilled funds.  In each figure, lambda is given in the title.  For each lambda, the first panel conducts the procedure so that the true alpha of the underlying distribution is 1\%.  The second and third use alpha =2.5\% and 5\%, respectively.  In the below, $\lambda=.5$.
\begin{center}
 \begin{figure}[h!]
 \includegraphics[scale=0.7]{plot3b.pdf}\label{fig:plot3b}
 \end{figure}
 \end{center}
\clearpage 
\textbf{Figure 10 : }Figures 7-10 show the distributions of estimated 95th (orange) and 5th (blue) percentile alpha T-statistics across bootstrap runs at each of four levels of lambda (.1, .25, .5, .75), the fraction of truly skilled funds.  In each figure, lambda is given in the title.  For each lambda, the first panel conducts the procedure so that the true alpha of the underlying distribution is 1\%.  The second and third use alpha =2.5\% and 5\%, respectively.  In the below, $\lambda=.75$.
\begin{center}
 \begin{figure}[h!]
 \includegraphics[scale=0.7]{plot4b.pdf}\label{fig:plot4b}
 \end{figure}
 \end{center}
\clearpage 
\end{document}
